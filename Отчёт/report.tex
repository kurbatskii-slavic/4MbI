\documentclass[a4paper,12pt,titlepage,final]{article}

\usepackage[T1,T2A]{fontenc}     % форматы шрифтов
\usepackage[utf8x]{inputenc}     % кодировка символов, используемая в данном файле
\usepackage{mathtools}
\usepackage[russian]{babel}      % пакет русификации
\usepackage{tikz}                % для создания иллюстраций
\usepackage{pgfplots}            % для вывода графиков функций
\usepackage{geometry}		 % для настройки размера полей
\usepackage{indentfirst}         % для отступа в первом абзаце секции
\usepackage{multirow}            % для таблицы с результатами
\usepackage{listings}
\lstset{extendedchars=\true}
% выбираем размер листа А4, все поля ставим по 3см
\geometry{a4paper,left=30mm,top=30mm,bottom=30mm,right=30mm}

\setcounter{secnumdepth}{3}      % включаем нумерацию секций и подсекций

\usepgfplotslibrary{fillbetween} % для изображения областей на графиках

\begin{document}
\begin{titlepage}
\centering\noindent
{
\begin{minipage}{0.1\textwidth}

\end{minipage}
\hfill
\begin{minipage}{0.77\textwidth}
\begin{center}
\textbf{МОСКОВСКИЙ ГОСУДАРСТВЕННЫЙ УНИВЕРСИТЕТ}\par
\textbf{имени М.В.Ломоносова}\par
\end{center}
\end{minipage}
\hfill
\begin{minipage}{0.1\textwidth}

\end{minipage}
}\par
{
\textbf{Факультет вычислительной математики и кибернетики}\par
\nointerlineskip
\noindent\makebox[\linewidth]{\rule{\textwidth}{0.4pt}}
}
\vfill
{
\Large{\textbf{Практическое задание по курсу лекций}}\par
\Large{\textbf{«Численные методы линейной алгебры»}}\par
}\\
{
\Large{\textbf{Задание №1}}\par
}
{
\Large{\textbf{Отчет}}\par
\Large{\textbf{о выполненном задании}}\par
\Large{студента 303 учебной группы факультета ВМК МГУ}\par
\Large{Курбацкого Вячеслава Константиновича}\par
}
\vfill
{\small Москва\\ \the\year{}}
\end{titlepage}
% Автоматически генерируем оглавление на отдельной странице
\tableofcontents
\newpage
\section{QR разложение невырожденной матрицы}
\subsection{Постановка задачи}
Требуется реализовать 2 метода получения QR разложения невырожденной матрицы A. Оба метода определены вариантом задания и были разобраны на лекциях. Также нужно сравнить точность и время работы каждого метода и решить СЛАУ с наилучшим из них, посчитав норму невязки решения и правой части.

\subsection{Описание методов}\\
\textbf{Определение.} \textit{QR разложением} матрицы $A \in \mathbb{R}^{n \times n}$ называется представление матрицы в виде: $$A = QR$$
Где $Q \in \mathbb{R}^{n \times n}$ - ортогональная матрица, а $R \in \mathbb{R}^{n \times n}$ - верхняя треугольная матрица. Наложение дополнительного условия неотрицательности диагональных элементов матрицы $R$ гарантирует единственность такого разложения. Так же на лекциях было доказано, что такое разложение существует для любой невырожденной матрицы $A$. Невырожденность матрицы $A$ гарантирует невырожденность матрицы $R$, т.к. $detA = detQ detR \Rightarrow detR = detA / detQ$. Это означает, что на диагонали треугольной матрицы $R$ нет нулевых элементов, т.к. $detR = r_{11}r_{22}\ldots r_{nn} \neq 0$. \\
В данной работе рассмотрены 2 метода получения $QR$ разложения: метод отражений (Хаусхолдера) и процесс ортогонализации Грама-Шмидта.
\subsubsection{Процесс ортогонализации Грама-Шмидта.}
Пусть имеется матрица $A \in \mathbb{R}^{n \times n}$, $detA \neq 0$. При условии невырожденности матрицы, её столбцы будут образовывать базис пространства $\mathbb{R}^n$ (будут являться линейно независимой системой векторов), поэтому к столбцам можно применить процесс ортогонализации Грама-Шмидта. Обозначим через $A^j$ - j-столбец матрицы $A$, через $Q^j$ - соответственно j-столбец матрицы $Q$. Тогда алгоритм будет выглядеть следующим образом:

$$ \begin{cases}
B^1 := A^1, \quad Q^1 = \frac{B^1}{\Vert B^1 \Vert_2} \\ \\
B^k := A^k - \sum_{s = 1}^{k-1} (A^k, Q^s) Q^s, \quad Q^k = \frac{B^k}{\Vert B^k \Vert_2}, \quad k = \overline{2,n}
\end{cases} $$
Нетрудно показать, что полученная система векторов $\{Q^j\}_{j=1}^{n}$ будет ортонормированной, т.е. вектора удовлетворяют соотношению:
$$(Q^i, Q^j) = \delta_{ij} = 
\begin{cases}
    1, \quad i = j \\
    0, \quad i \neq j
\end{cases}
$$
Таким образов, собрав полученные вектора в столбцы матрицы $Q$ получится матрица: $QQ^T = Q^TQ = I$ - ортогональная матрица. Так же легко заметить, что $Q^k$ выражается через первые k столбцов матрицы $A$, поэтому матрица коэффициентов $R = (r_{ij})$, где $r_{ij} = (Q^i, A^j)$ при $i \neq j$ и $r_{ii} = \Vert B^i \Vert_2 > 0$ будет верхней треугольной и удовлетворять соотношению $A = QR$ - искомое $QR$ разложение. \\ \\
На лекции было показано, что вычислительная сложность процесса ортогонализации составляет $n^3 + O(n^2)$. Данный алгоритм прост в реализации, но требует относительно много вычислений, по сравнению, например, с методом отражений.
\subsubsection{Метод отражений (Хаусхолдера)}
\textbf{Определение.} Матрицей отражений (матрицей Хаусхолдера) называется матрица вида $$Z = I - 2 ww^T,$$ где $I$ - единичная матрица и $\Vert w \Vert_2 = 1$. Данная матрица обладает рядом полезных на практике свойств:
\begin{itemize}
    \item $Z = Z^* = Z^{-1}$ (то есть матрица ортогональна)
    \item $detZ = -1$
    \item $\lambda_1 = 1$, $\lambda_i = -1$ при $i = \overline{2, n}$ - собственные значения матрицы.
\end{itemize}
Линейный оператор, матрица которого в единичном базисе $e_1, e_2 \ldots e_n$ является матрицей отражений, называется оператором отражений. В силу инвариантности относительно подобия, выше указанные свойства сохраняются в любом базисе, а следовательно являются свойствами самого оператора отражений. \\ \\
На лекции был доказан следующий факт: \\ \\
\textbf{Утверждение.} Пусть $x$ и $y$ - ненулевые векторы, причем $\Vert x \Vert_2 = \Vert y \Vert_2$. Тогда если $$w = \frac{x - y}{\Vert x - y\Vert_2} \text{ или } w = -\frac{x - y}{\Vert x - y\Vert_2},$$
то заданный этим вектором оператор отражения будет таким, что $y = Zx$. \\ На основе этого утверждения строится ещё один метод нахождения $QR$ разложения матрицы - метод отражений (Хаусхолдера): \\ \\

Пусть 
$$A^{(0)} = \begin{bmatrix}
a_{11}^{(0)} & a_{12}^{(0)} & a_{13}^{(0)} & \ldots & a_{1n}^{(0)} \\
a_{21}^{(0)} & a_{22}^{(0)} & a_{23}^{(0)} & \ldots & a_{2n}^{(0)} \\
a_{31}^{(0)} & a_{32}^{(0)} & a_{33}^{(0)} & \ldots & a_{3n}^{(0)} \\
\vdots & \vdots & \vdots & \ddots & \vdots \\
a_{n1}^{(0)} & a_{n2}^{(0)} & a_{n3}^{(0)} & \ldots & a_{nn}^{(0)} \\
\end{bmatrix} $$
и пусть $x^{(1)} = \begin{pmatrix} a_{11}^{(0)} & a_{21}^{(0)} & \ldots & a_{n1}^{(0)}\end{pmatrix}^T$ - первый столбец матрицы $A^{(0)}$ и $y^{(1)} = \begin{pmatrix} \Vert x^{(1)} \Vert_2 & 0 & \ldots & 0\end{pmatrix}^T$. Эти векторы имеют одинаковую длину, поэтому согласно утверждению можно построить матрицу $Z^{(1)}$ такую, что: $Z^{(1)}x^{(1)} = y^{(1)}$. Тогда:
$$A^{(1)} = Z^{(1)}A^{(0)} = 
\begin{bmatrix}
\Vert x^{(1)} \Vert_2 & a_{12}^{(1)} & a_{13}^{(1)} & \ldots & a_{1n}^{(1)} \\
0 & a_{22}^{(1)} & a_{23}^{(1)} & \ldots & a_{2n}^{(1)} \\
0 & a_{32}^{(1)} & a_{33}^{(1)} & \ldots & a_{3n}^{(1)} \\
\vdots & \vdots & \vdots & \ddots & \vdots \\
0 & a_{n2}^{(1)} & a_{n3}^{(1)} & \ldots & a_{nn}^{(1)} \\
\end{bmatrix}$$
В общем случае после исключения $x_{k-1}$ из уравнений $k$, $k+1$, \ldots $n$ имеем матрицу:
$$ A^{(k-1)} = \begin{bmatrix}
    R^{(k-1)} & B \\
    \Theta & \Tilde{A}^{(k-1)}
\end{bmatrix}
$$
где $R^{(k-1)}$ - верхняя треугольная размера $(k-1)\times(k-1)$, $\Theta$ - нулевая и $$\Tilde{A}^{(k-1)} = \begin{bmatrix}
    a_{kk}^{(k-1)} & a_{kk+1}^{(k-1)} & \ldots & a_{kn}^{(k-1)} \\
    a_{k+1k}^{(k-1)} & a_{k+1k+1}^{(k-1)} & \ldots & a_{k+1n}^{(k-1)} \\
    \vdots & \vdots & \ddots & \vdots \\
    a_{nk}^{(k-1)} & a_{nk+1}^{(k-1)} & \ldots & a_{nn}^{(k-1)} \\
\end{bmatrix}
$$
Тогда возьмем  $x^{(k)} = \begin{pmatrix} a_{kk}^{(k-1)} & a_{k+1k}^{(k-1)} & \ldots & a_{nk}^{(k-1)}\end{pmatrix}^T \in \mathbb{R}^{n-k+1}$ и  $y^{(k)} = \begin{pmatrix} \Vert x^{(k)} \Vert_2 & 0 & \ldots & 0\end{pmatrix}^T \in \mathbb{R}^{n-k+1}$. Возьмём $w^{(k)}$ как в утверждении и построим матрицу:
$$Z^{(k)} = \begin{bmatrix}
    I_{k-1} & \Theta \\
    \Theta & I_{n - k + 1} - 2w^{(k)}w^{(k)^T} \\
\end{bmatrix} $$
В данных обозначениях матрица $I_j$ - единичная матрица размера $j \times j$. $\Theta$ - нулевые блоки нужных размеров. В таком случае:
$$A^{(k)} = Z^{(k)}A^{(k-1)} = \begin{bmatrix}
    R^{(k-1)} & B \\
    \Theta & \Tilde{A}^{(k)}
\end{bmatrix}
$$
где $$
\Tilde{A}^{(k)} = \begin{bmatrix}
    \Vert x^{(k)} \Vert_2 & a_{kk+1}^{(k)} & \ldots & a_{kn}^{(k)} \\
    0 & a_{k+1k+1}^{(k)} & \ldots & a_{k+1n}^{(k)} \\
    \vdots & \vdots & \ddots & \vdots \\
    0 & a_{nk+1}^{(k)} & \ldots & a_{nn}^{(k)} \\
\end{bmatrix}
$$
Применяя эти рассуждения при всех $k = \overline{1, n-1}$ получаем:
$$R = Z^{(n-1)}Z^{(n-2)}\ldots Z^{(1)}A = Q^TA \Leftrightarrow A = QR$$
\textbf{Замечание.} Если вектора $x$ и $y$ близки друг к другу, то есть величина $\Vert x - y \Vert_2 << 1$, то при вычислении вектора $w$ происходит деление на очень маленькое число с плавающей точкой, что может приводить к большим ошибкам округления. В таком случае рекомендуется отражать $x \to -y$ с добавлением коэффициента -1 к матрице $Z$. Тогда $w = \frac{x + y}{\Vert x + y \Vert_2}$ и $Z = -(I - 2ww^T)$. Такое преобразование приведет к такому же результату, но к тому же помогает избегать деления на очень маленькое число с плавающей точкой. На практике удобно определять вид матрицы по знаку скалярного произведения: $(x, y) < 0 \Rightarrow $ угол между векторами тупой, поэтому допустимо брать $x - y$. В случае острого угла следует брать $x + y$. Именно такой подход и реализован в написанной программе.\\ \\
На лекции было показано, что сложность вычислений данного метода составляет $\frac{2}{3}n^3 + O(n^2)$. Данный метод требует меньше вычислений, чем процесс ортогонализации Грама-Шмидта, но он более труден в реализации, поскольку необходимо реализовать уменьшение размерности векторов с каждым шагом (эта трудность относится именно к программной реализации, а не аналитическому нахождению разложения). Кроме того, нужно следить за делением на близкие к нулю числа с плавающей точкой - тоже дополнительное усложнее и уменьшение надежности.
\subsection{Программная реализация}
Заданием было предложено реализовать упомянутые методы на языке программирования С (С++). Для реализации были использованы контейнеры $std::vector$ и написанные на их основе классы для матриц. Для уменьшения влияния чисто программных недочётов были использованы флаги компиляции $-O3 -march=native$ для максимальной оптимизации. 
\subsubsection{Время, потраченное на построение и погрешность вычеслений}
Для усреднения полученных значений проГрама была запущена 1000 раз (каждый раз вектор $x$ генерировался случайно, $x_k \in [-1, 1]$ - согласно условию задания). В качестве матричной нормы используется норма, подчиненная максимум-норме: $\Vert x \Vert = \underset{i=1,2\ldots n}{\max}|x_i|$. В следующей таблице представлены полученные значения: 
$$
\begin{tabular}{|c|c|c|}
\hline 
Метод & Среднее время (мс) & $\Vert A - QR\Vert$ \\
\hline
     Хаусхолдер & 0.681552 & 9.05193e-13 \\
\hline
     Грам-Шмидт & 0.872347 & 1.30826e-13 \\
\hline
\end{tabular}
$$
Хоть по теоретическим оценкам можно было ожидать, что метод Хаусхолдера в среднем будет в 1.5 раза быстрее, но программная реализация накладывает свои особенности, такие как работа с памятью и оптимизация вычислений, поэтому полученные на практике значения могут немного расходиться с теоретическими. Тем не менее даже на достаточно небольшой матрице размера 100 (матрица соответствует варианту задания) можно заметить, что метод Хаусхолдера работает быстрее процесса ортогонализации, но в то же время даёт точность почти на порядок хуже. Это можно объяснить как и особенностями конкретной реализации, так и тем, что метод Хаусхолдера, как известно, не самый надежный в плане вычислений.
\section{Решение СЛАУ}
В задании предложено решить СЛАУ с наиболее точным методом получения $QR$ разложения. Но для начала стоит напомнить, как решать системы линейных алгебраических уравнений $Ax = f$ при помощи $QR$ разложения матрицы $A$.
\subsection{Обратный ход метода Гаусса}
$$Ax = f \Leftrightarrow QRx = f \Leftrightarrow Rx = Q^Tf = \Tilde{f}$$
Таким образом, система сводится к системе с верхней треугольной матрицей $R$. Решение такой системы получается через обратный ход метода Гаусса: \\
$$ \begin{cases}
x_n = \Tilde{f}_n / r_{nn} \\ \\
x_m = [\Tilde{f}_m - \sum_{j=m+1}^n r_{mj}x_j]/r_{mm}. \quad m = \overline{n-1,1}
\end{cases} $$
Как было упомянуто в начале, в силу невырожденности $A$, все $r_{ii} \neq 0 $ при $i = \overline{1, n}$, поэтому все арифметические операции корректны. Пользуясь этими соотношениями, решим систему $Ax = f$ при помощи процессом ортогонализации Грама-Шмидта.
\subsection{Результаты вычислений. Невязка.}
Как и в случае вычисления $\Vert A - QR \Vert$ здесь испольузется максимум норма.
$$
\begin{tabular}{|c|c|c|c|}
\hline 
Метод & $\Vert x - \Tilde{x} \Vert$ & $\Vert f - A\Tilde{x}\Vert$ & Время (мс)\\
\hline
     Грам-Шмидт & 1.66533e-15 & 1.13687e-13 & 0.005775 \\
\hline
\end{tabular}
$$
Точность нахождения решения довольно неплохая. Однако при увеличении размера матрицы точность вычислений будет падать на порядки и станет хуже, чем у метода Хаусхолдера (эти данные не представлены в отчёте, т.к. это не просится в задании, при желании можно запустить программу на тестах с матрицами большей размерности).
\newpage
\section{Выводы.}
Были реализованы и сравнены 2 метода построения $QR$ разложения невырожденной матрицы. Для каждого решения были оценены нормы разности $A - QR$ и для наиболее точного метода была решена система $Ax = f$ и оценена норма невязок $x - \Tilde{x}$ и $f - A\Tilde{x}$, где $\Tilde{x}$ - численное решение системы, полученное реализацией обратного хода метода Гаусса, а $x$ - вектор, такой что $f = Ax$. В следующем разделе приведены инструкции для компиляции и запуска программы. 
\newpage
\section{Инструкции по запуску программы.}
\begin{itemize}
    \item Запустить скрипт run.sh командой 
\begin{lstlisting}
$ ./run.sh 
\end{lstlisting} 
Данный скрипт создаст директорию build и внутри неё скомпилирует проект с помощью cmake и Makefile. 
    \item находясь в директории build команда 
\begin{lstlisting}
$ ./main 100 < ../tests/SLAU_var_6.txt
\end{lstlisting} 
    запустит программу на матрице размера $100 \times 100$, заданной вариантом. Кроме того, предусмотрена возможность тестирования на матрицах размера 1024, 2048 и 4096:
\begin{lstlisting}
$ ./main 1024 < ../tests/in1024.txt
$ ./main 2048 < ../tests/in2048.txt
$ ./main 4096 < ../tests/in4096.txt
\end{lstlisting} 
Во всех случаях вывод будет примерно такой:
\begin{lstlisting}
Time (Householder): 0.654763ms
||A - QR|| = 8.95672e-13

Time: (Gram-Shmidt): 0.860597ms
||A - QR|| = 1.30826e-13

Time: (Householder system): 0.021012ms
||x - x_f|| = 3.44169e-15
||f - Ax_f|| = 2.20268e-13

Time: (Gram-Shmidt system): 0.026521ms
||x - x_f|| = 1.88738e-15
||f - Ax_f|| = 1.35003e-13
\end{lstlisting} 
\end{itemize}
\end{document}
